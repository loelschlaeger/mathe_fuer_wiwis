\documentclass[aspectratio=169]{beamer}

\usepackage{array}
\usepackage{tikz}
\setbeamertemplate{navigation symbols}{}

\newcommand{\countdown}[1]{
	\transduration{0.2}
	\foreach \x in {0,...,#1}{\only<+>{
		\pgfdeclarehorizontalshading{myshade}{1em}{%
			color(0\textwidth)=(rgb:green,#1;-green,\x;red,\x);
			color(\x*(1/#1)*\textwidth)=(rgb:green,#1;-green,\x;red,\x);     
			color((1/#1)*\textwidth+\x*(1/#1)*\textwidth)=(white);
			color(\textwidth)=(white)
		}
		\begin{pgfpicture}{0pt}{0pt}{\textwidth}{2em}
			\pgfpathrectangle{\pgfpointorigin}{\pgfpoint{\textwidth}{2em}}
			\pgfusepath{clip}
			\pgftext[left,base]{\pgfuseshading{myshade}}
		\end{pgfpicture}
	}}
}

\newcommand{\question}[2]{
	\vfill
	\center{Frage \insertframenumber\ von 10}
	\vspace{0.5cm}
	{\LARGE \center #1}
	\vfill
	\countdown{#2}
	\vfill
}			

\begin{document}

\begin{frame}
	\vfill
	\begin{center}
		\LARGE Bereit?
	\end{center}
	\pause
	\vfill
	\countdown{20}
	\vfill
\end{frame}

\addtocounter{framenumber}{-1}

\frame{\question{Wie viel ist $6 : 2(1+2)$~?}{100}}
\frame{\question{Gilt $\displaystyle (a+b)(a-b) = a^2 - b^2$~?}{100}}
\frame{\question{Wie viel ist $\displaystyle \frac{3:4}{4:3}$~?}{100}}
\frame{\question{Ist $\displaystyle \frac27$ größer als $\displaystyle \frac39$~?}{100}}
\frame{\question{Wie viel ist $5\%$ von $10\%$~?}{100}}
\frame{\question{Was ist der Definitionsbereich von $\displaystyle f(x) = \frac{1}{3+x}$~?}{100}}
\frame{\question{Gilt $\displaystyle 3^{-2} = \frac19$~?}{100}}
\frame{\question{Wie viele Lösungen hat $|x - 1| = 2$~?}{100}}
\frame{\question{Wie groß ist $\displaystyle \int_{-1}^{1} x\,dx$~?}{100}}
\frame{\question{Wie viel ist $\displaystyle \sum_{i=1}^3 i$~?}{100}}

\begin{frame}
\pause
\renewcommand{\arraystretch}{1.5}
\begin{tabular}{rll}
	\visible<+->{1} & \visible<+->{Wie viel ist $6 : 2(1+2)$?} & \visible<+->{$6 : 2(1+2) = 6 : 2 \cdot 3 = 3 \cdot 3 = 9$} \\
	\visible<+->{2} & \visible<+->{Gilt $(a+b)(a-b) = a^2 - b^2$?} & \visible<+->{Ja (dritte binomische Formel)} \\
	\visible<+->{3} & \visible<+->{Wie viel ist $\frac{3:4}{4:3}$?} & \visible<+->{$\frac{3:4}{4:3} = \frac{\frac34}{\frac43} = \frac34 \cdot \frac34 = \frac{9}{16}$} \\
	\visible<+->{4} & \visible<+->{Ist $\frac27$ größer als $\frac39$?} & \visible<+->{$\frac27 = \frac{18}{63} < \frac{21}{63} = \frac39$} \\
	\visible<+->{5} & \visible<+->{Wie viel ist $5\%$ von $10\%$?} & \visible<+->{$0.05 \cdot 0.1 = 0.005 = 0.5\%$} \\
	\visible<+->{6} & \visible<+->{[...] Definitionsbereich von $f(x) = \frac{1}{3+x}$?} & \visible<+->{Reelle Zahlen außer $-3$} \\
	\visible<+->{7} & \visible<+->{Gilt $3^{-2} = \frac19$?} & \visible<+->{Ja, denn $3^{-2} = \frac{1}{3^2} = \frac19$} \\
	\visible<+->{8} & \visible<+->{Wie viele Lösungen hat $|x - 1| = 2$?} & \visible<+->{Zwei ($x=3$ und $x=-1$)} \\
	\visible<+->{9} & \visible<+->{Wie groß ist $\int_{-1}^{1} x\,dx$?} & \visible<+->{$0$} \\
	\visible<+->{10} & \visible<+->{Wie viel ist $\sum_{i=1}^3 i$?} & \visible<+->{$1+2+3=6$} \\
\end{tabular}
\end{frame}

\end{document}