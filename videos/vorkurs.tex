\documentclass[aspectratio=169]{beamer}

\usepackage{tikz}
\setbeamertemplate{navigation symbols}{}

\newcommand{\countdown}[1]{
	\transduration{0.2}
	\foreach \x in {0,...,#1}{\only<+>{
		\pgfdeclarehorizontalshading{myshade}{1em}{%
			color(0\textwidth)=(rgb:green,#1;-green,\x;red,\x);
			color(\x*(1/#1)*\textwidth)=(rgb:green,#1;-green,\x;red,\x);     
			color((1/#1)*\textwidth+\x*(1/#1)*\textwidth)=(white);
			color(\textwidth)=(white)
		}
		\begin{pgfpicture}{0pt}{0pt}{\textwidth}{2em}
			\pgfpathrectangle{\pgfpointorigin}{\pgfpoint{\textwidth}{2em}}
			\pgfusepath{clip}
			\pgftext[left,base]{\pgfuseshading{myshade}}
		\end{pgfpicture}
	}}
}

\newcommand{\question}[2]{
	\vfill
	\center{Frage \insertframenumber\ von \inserttotalframenumber}
	\vspace{0.5cm}
	{\LARGE \center #1}
	\vfill
	\countdown{#2}
	\vfill
}			

\begin{document}

\begin{frame}
	\vfill
	\begin{center}
		\LARGE Bereit?
	\end{center}
	\vfill
	\countdown{20}
	\vfill
\end{frame}
\addtocounter{framenumber}{-1}
\frame{\question{Wie viel ist $6 : 2(1+2)$~?}{100}}
\frame{\question{Wie nennt man das Gesetz $a + (b + c) = (a + b) + c$~?}{100}}
\frame{\question{Ist $\displaystyle \frac27$ größer als $\displaystyle \frac39$~?}{100}}
\frame{\question{Wie viel ist $\displaystyle \frac{3:4}{4:3}$~?}{100}}
\frame{\question{Gilt $\displaystyle 3^{-2} = \frac19$~?}{100}}
\frame{\question{Wie viel ist $5\%$ von $10\%$~?}{100}}
\frame{\question{Was ist der Definitionsbereich von $\displaystyle f(x) = \frac{1}{3+x}$~?}{100}}
\frame{\question{Wie viele Lösungen hat $|x - 1| = 2$~?}{100}}
\frame{\question{Gilt $\displaystyle (a+b)(a-b) = a^2 - b^2$~?}{100}}
\frame{\question{Wie viel ist $\displaystyle \sum_{i=1}^3 i$~?}{100}}

\end{document}